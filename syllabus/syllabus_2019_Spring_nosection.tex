\documentclass[12pt,letterpaper]{article}
\usepackage[utf8x]{inputenc}
\usepackage[left=2cm,right=2cm,top=2cm,bottom=2cm]{geometry}
\usepackage{lastpage}
\usepackage{fancyhdr}
\usepackage{newtxtext,newtxmath}
\usepackage{lastpage}

\pagestyle{fancy}
\fancyhf{}
\lhead{\textsc{BHCC Mat-181}}
\chead{\textsc{Syllabus}}
\rhead{\textsc{Fall 2019}}
\rfoot{Page \thepage ~of \pageref{LastPage}}


\usepackage{termcal}
\newcommand{\bs}{\bf\sffamily}
\renewcommand{\calprintdate}{\it \monthname\ \ordinaldate}
\renewcommand{\calprintclass}{\textbf{\small \#\theclassnum}}

\newcommand{\MTWRClass}{%
\calday[Monday]{\classday} % Monday (no class)
\calday[Tuesday]{\classday} % Tuesday
\calday[Wednesday]{\classday}
\calday[Thursday]{\classday} % Thursday
\skipday % Friday 
\skipday\skipday % weekend (no class)
}
\newcommand{\Holiday}[2]{%
\options{#1}{\noclassday}
\caltext{#1}{#2}
}

\begin{document}

\subsection*{Professor}
\begin{itemize}
\item Chad Worley 
\item ceworley@bhcc.mass.edu
\end{itemize}
\subsection*{Section}
\begin{itemize}
\item MAT-098 / MAT-181
\item ---------
\item ---------
\item ---------
\end{itemize}


\subsection*{Class Materials}
\begin{itemize}
\item OpenIntro Statistics 4th Edition
\\ www.openintro.com
\\ The textbook is free to download. You can buy paper copies for about \$15 on Amazon. I also have copies to lend.
\item A scientific calculator is needed. I recommend the TI-36x Pro (\$25 online). With a smart phone, you can use Geogebra Scientific Calculator. You will need to put your phone on airplane mode to use this during tests. If you cannot afford a calculator and do not have a smart phone, please send me an email for assistance.
\end{itemize}

\subsection*{Important Dates}
\begin{itemize}
\item Tests: Feb 21, Mar 28, Apr 25
\item Final exam: May 16
\item Holidays: Feb 18, Mar 18-21, Apr 15
\item Project proposal: Mar 14
\item Project paper: May 9
\item Project presentations: May 8-9
\end{itemize}

\subsection*{Grade Weights}
\begin{itemize}
\item Tests (30\%)
\item Final exam (20\%)
\item Homework (20\%)
\item Attendence/in-class work/participation (20\%)
\item Project (10\%)
\end{itemize}


\newpage
\subsection*{Schedule}
We will likely stick to this schedule very closely. If there is a snow day, you will watch that chapter's video (and/or read the chapter) and complete the exercises. You should expect to spend at least 5 hours on homework each week. 

Most days have the chapter we will discuss and the homework exercises. The homework exercises are at the end of the given chapter. The homework is due the following class.

\begin{center}
\sffamily %\large
\begin{calendar}{1/21/2019}{17} % Semester starts on 1/11/2010 and last for 16
                    % weeks, including finals week
\setlength{\calboxdepth}{.4in}
\setlength{\calwidth}{\textwidth}
\MTWRClass
\caltexton{1}{Ch 1.1\\  Ex 1.1-1.2}
\caltextnext{Ch 1.2\\ Ex 1.3-1.8}
\caltextnext{Ch 1.3\\ Ex 1.9-1.16}
\caltextnext{ Ch 1.4\\ Ex 1.17-1.29}
\caltextnext{ Ch 1.5\\ Ex 1.30-1.37}
\caltextnext{ Ch 1.6\\ Ex 1.38-1.47}
\caltextnext{ Ch 1.6\\ Ex 1.48-1.64}
\caltextnext{ Ch 1.7\\ Ex 1.65-1.68}
\caltextnext{ Ch 1.8\\ Ex 1.69-1.70}
\caltextnext{ Ch 2.1\\ Ex 2.1-2.8}
\caltextnext{ Ch 2.1\\ Ex 2.9-2.14}
\caltextnext{ Ch 2.2\\ Ex 2.15-2.21}
\caltextnext{ Ch 2.2\\ Ex 2.22-2.26}
\caltextnext{ Ch 2.3-2.4\\ Ex 2.27-2.35}
\caltextnext{ Ch 2.5\\ Ex 2.36-2.44}
\caltextnext{ Review \\  Review}
\caltextnext{ Review \\  Review}
\caltextnext{{\bf \sffamily TEST}}
\caltextnext{ Ch 3.1\\ Ex 3.1-3.8}
\caltextnext{ Ch 3.1\\ Ex 3.9-3.16}
\caltextnext{ Ch 3.2\\ Ex 3.17-3.18}
\caltextnext{ Ch 3.3\\ Ex 3.19-3.24}
\caltextnext{ Ch 3.4\\ Ex 3.25-3.30}
\caltextnext{ Ch 3.4\\ Ex 3.31-3.36}
\caltextnext{ Ch 4.1\\ Ex 4.1-4.6}
\caltextnext{ Ch 4.2\\ Ex 4.7-4.16}
\caltextnext{ Ch 4.3\\ Ex 4.17-4.23}
\caltextnext{ Ch 4.3\\ Ex 4.24-4.32}
\caltextnext{ Ch 4.4\\ Ex 4.33-4.42}
\caltextnext{ Ch 4.5\\ Ex 4.43-4.48}
\caltextnext{ Review \\  Review}
\caltextnext{ Review \\  Review}
\caltextnext{{\bf \sffamily TEST}}

\caltextnext{ Ch 5.1\\ Ex 5.1-5.7}
\caltextnext{ Ch 5.1\\ Ex 5.8-3.14}
\caltextnext{ Ch 5.2\\ Ex 5.15-3.24}
\caltextnext{ Ch 5.3\\ Ex 5.25-3.31}
\caltextnext{ Ch 5.3\\ Ex 5.32-3.38}
\caltextnext{ Ch 5.3\\ TBD}
\caltextnext{ Ch 5.4\\ Ex 5.39-5.14}
\caltextnext{ Ch 5.5\\ Ex 5.41-5.52}
\caltextnext{ Ch 6.1\\ Ex 6.1-6.11}
\caltextnext{ Ch 6.1\\ Ex 6.12-6.22}
\caltextnext{ Ch 6.2\\ Ex 6.23-4.30}
\caltextnext{ Ch 6.2\\ Ex 6.31-4.38}
\caltextnext{ Ch 6.5-6.6\\ Ex 6.51-6.56}
\caltextnext{ Review \\  Review}
\caltextnext{{\bf \sffamily TEST}}
\caltextnext{ Ch 7.1\\ Ex 7.1-7.9}
\caltextnext{ Ch 7.1\\ Ex 7.10-7.18}
\caltextnext{ Ch 7.2\\ Ex 7.19-3.30}
\caltextnext{ Ch 7.3\\ Ex 7.31-3.34}
\caltextnext{ Ch 7.4\\ Ex 7.35-3.44}
\caltextnext{TBD}
\caltextnext{Presentations}
\caltextnext{Presentations}
\caltextnext{ Review \\  Review}
\caltextnext{ Review \\  Review}
\caltextnext{ Review \\  Review}
\caltextnext{\bs Final Exam}


% Holidays
\Holiday{1/21/2019}{\bs Martin Luther King Jr. Day}
\Holiday{2/18/2019}{\bs Presidents Day}
\Holiday{3/18/2019}{\bs Spring Break}
\Holiday{3/19/2019}{\bs Spring Break}
\Holiday{3/20/2019}{\bs Spring Break}
\Holiday{3/21/2019}{\bs Spring Break}
\Holiday{4/15/2019}{\bs Patriots Day}
\Holiday{3/4/2019}{\bs Snow Day}


\caltext{3/14/2019}{\textbf{Proposal Due}}
\caltext{5/9/2019}{\textbf{Paper Due}}
\end{calendar}
\end{center}


\subsection*{Course description}
This course covers statistical concepts and methods.  Topics include being able to summarize and analyze data distributions both numerically (averages and variation) and graphically.  Evaluating linear equations while understanding the concepts of slope, intercepts, inequalities, correlation and regression will be discussed.  The concept of probability and probability distributions will be introduced for both discrete and continuous variables.  Other topics include:  binomial, normal, and t-distributions; estimation and hypothesis testing.  This course meets General Education “Quantitative Thought” Requirement Area 4.
Prerequisite: A grade of C or better in Foundations of Mathematics (MAT093) or placement. Credit hours: 6

\subsection*{Instructional Objectives}
\begin{enumerate}
\item Identify types of data
\item Identify the measurement level of a variable
\item Identify basic sampling techniques
\item Organize data using frequency distributions
\item Represent frequency distributions graphically
\item Represent data using bar graphs
\item Summarize data using mean, median, and mode
\item Describe data using range, variance, and standard deviation
\item Identify the position of a data point by using percentiles and standard scores
\item Produce stem and leaf displays and box and whisker plots
\item Determine the number of possible outcomes using a tree diagram
\item Find the total number of possible outcomes using the multiplication rule
\item Calculate the number of permutations of $n$ things taken $r$ at a time
\item Calculate the number of combinations of $n$ things taken $r$ at a time
\item Determine sample spaces
\item Find the probability of an event using relative frequencies
\item Find the probability of a compound event
\item Find the conditional probability of an event
\item Construct a probability distribution for a discrete random variable
\item Find the expected value and standard deviation for a discrete random variable
\item Calculate binomial probabilities
\item Find the mean and standard deviation for a binomial distribution
\item Identify the properties of a normal distribution
\item Find the area under the standard normal distribution for various intervals
\item Transform a normally distributed random variable into a standard normal variable
\item Find specific data values for given areas under a normal distribution
\item State the Central Limit Theorem
\item Use the Central Limit Theorem to solve problems involving the distribution of the sample mean for large samples
\item Use the normal distribution to approximate probabilities for a binomial
\item Distinguish between point estimates and interval estimates
\item Find the confidence interval for $\mu$ with $\sigma$ known
\item Find the confidence interval for $\mu$ with $\sigma$ unknown
\item Structure a classical test of hypothesis
\item Test means for one-sample (using large and small samples)
\item Test for a proportion
\item Test the difference between means for dependent samples
\item Find the equation of the least squares regression line
\item Compute the standard error of the estimate
\item Find the confidence interval for the dependent variable
\item Compute the linear correlation coefficient
\item Test for a significant linear correlation
\item Compute the coefficient of determination
\end{enumerate}

\subsection*{Project}
You will have a semester-long project. You will perform an experiment (RCT) and report the results in a paper and presentation. The proposal is due before spring break. The paper will follow the abstract-background-method-result-discussion format.

\subsection*{Student code of conduct}
All students are expected to adhere to the honor code regarding course assignments and exams.  This includes completing assignments without unauthorized aid when instructed.  Any student suspected of cheating on an assignment or an exam will not pass that assignment or exam.

\subsection*{Individuals with a disability}
Bunker Hill Community College is committed to providing equal access to the educational experience for all students in compliance with Section 504 of the Rehabilitation Act of 1973 and the Americans with Disabilities Act of 1990

\subsection*{College Accommodations and Support Services }
The Office of Disability Support Services is a student-focused department dedicated to assisting members of the BHCC community with documented physical and/or learning disabilities. Students may be eligible for services that include tutoring, testing and other classroom accommodations. To get more information or request an accommodation, contact the Disability Support Services Office at 617-228-2327 (Room E222). Students are encouraged to request accommodations as early as possible and ideally before the start of the semester. For information about programs and services please visit: http://www.bhcc.mass.edu/disabilitysupportservices/.  


\end{document}
